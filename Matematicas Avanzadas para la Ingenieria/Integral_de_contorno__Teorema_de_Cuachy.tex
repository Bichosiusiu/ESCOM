\documentclass{article}
\usepackage{amsfonts}
\usepackage{amsmath}
\usepackage{graphicx} % Required for inserting images
\usepackage{amsmath, mathtools}
\mathtoolsset{showonlyrefs=true}

\begin{document}

%Portada
\begin{titlepage}
\centering
{\bfseries\LARGE Escuela Superior de Cómputo \par}
\vspace{1cm}
{\scshape\Large Ingeniería en Sistemas Computacionales \par}
\vspace{3cm}
{\scshape\Huge Matemáticas avanzadas para la ingeniería \par}
\vspace{1cm}
{\Large \textsc{Integral de contorno- Teorema de Cuachy} \par}
\vspace{3cm}
{\itshape\Large Grupo: 4CV2 \par}
{\itshape\Large Profesor: Zárate Cárdenas Alejandro \par}
\vfill
{\Large Equipo: \par}
{\Large Arellano Millán Gabriel \par}
{\Large Gómez Tovar Yoshua Oziel \par}
{\Large Herrera Tovar Karla Elena \par}
{\Large Vázquez Blancas César Said \par}
{\Large Zarco Sosa Kevin \par}
\vfill
{\Large 6 de junio de 2024 \par}
\end{titlepage}
\section{Ejercicio 1}
\[
\oint_{C}  \, z^2dz \quad con \quad C=-1,1,1+i,-1+i
\]
\(1)\quad con \quad C= -1 \) 
\begin{equation}
z_1=-1+(1+1)t
\end{equation}
\begin{equation}
z_1=-1+2t \quad 0\leq t \leq 1
\end{equation}
\begin{equation}
z^2=4t^2-4t+1
\end{equation}
\begin{equation}
dz=2dt
\end{equation}
\begin{equation}
\int_{0}^{1} 4t^2-4t+1(2dt)
\end{equation}
\begin{equation}
8\int_{0}^{1} t^2dt -8\int_{0}^{1} tdt + 2\int_{0}^{1}dt
\end{equation}
\begin{equation}
\left.\frac{8t^3}{3}-4t^2+2t \right|_{0}^{1}
\end{equation}
\begin{equation}
\frac{8}{3}-4+2+0= \frac{8}{3}-2=\frac{2}{3}
\end{equation}



\(2) \quad con\quad C= 1 \) 
\begin{equation}
z_2=1+(1+i-1)t
\end{equation}
\begin{equation}
z_2=-1+ti \quad 0\leq t \leq 1
\end{equation}
\begin{equation}
z^2=1+2ti-t^2
\end{equation}
\begin{equation}
dz=i
\end{equation}
\begin{equation}
\int_{0}^{1} (1+2ti-t^2)idt
\end{equation}
\begin{equation}
\int_{0}^{1} idt -2\int_{0}^{1} tdt -i\int_{0}^{1}t^2dt
\end{equation}
\begin{equation}
\left.it-t^2-\frac{it^3}{3} \right|_{0}^{1}
\end{equation}
\begin{equation}
it-t^2-\frac{i}{3}-0=-1+\frac{2}{3}i
\end{equation}


\(3)\quad con\quad C= 1+i \) 
\begin{equation}
z_3=1+i+(-1+i-1-i)t
\end{equation}
\begin{equation}
z_3=1+i-2t \quad 0\leq t \leq 1
\end{equation}
\begin{equation}
z^2=4t^2-4t+2i-4ti
\end{equation}
\begin{equation}
dz=-2
\end{equation}
\begin{equation}
\int_{0}^{1} 4t^2-4t+2i-4ti(-2)dt
\end{equation}
\begin{equation}
-8\int_{0}^{1} t^2dt +8\int_{0}^{1} tdt -4i\int_{0}^{1}dt+8i\int_{0}^{1}tdt
\end{equation}
\begin{equation}
\left.-\frac{8t^3}{3} +4t^2-4it+4t^2i\right|_{0}^{1}
\end{equation}
\begin{equation}
-\frac{8}{3}+4-4i+4i=\frac{4}{3}
\end{equation}



\(4) \quad con \quad C= -1+i \) 
\begin{equation}
z_4=-1+i+(-1+1-i)t
\end{equation}
\begin{equation}
z_4=-1+i-ti \quad 0\leq t \leq 1
\end{equation}
\begin{equation}
z^2=-t^2-2t-2i+2ti
\end{equation}
\begin{equation}
dz=-i
\end{equation}
\begin{equation}
\int_{0}^{1} -t^2+2t-2i+2ti(-i)dt
\end{equation}
\begin{equation}
i\int_{0}^{1} t^2dt -2i\int_{0}^{1} tdt -2\int_{0}^{1}dt+2\int_{0}^{1}tdt
\end{equation}
\begin{equation}
\left.\frac{it^3}{3} -it^2-2t+t^2\right|_{0}^{1}
\end{equation}
\begin{equation}
\frac{i}{3}-i-2+1=-\frac{2i}{3}-1
\end{equation}
Entonces sumando los resultados
\begin{equation}
-\frac{2i}{3}-1+\frac{4}{3}-1+\frac{2i}{3}+\frac{2}{3}
\end{equation}
\begin{equation}
-2+\frac{6}{3}=-2+2=0
\end{equation}
El teorema de cauchy dice que la funcion holomorfa es 0, por lo tanto el teorema está comprobado


\section{Ejercicio 3}
Comprobar el resultado del ejemplo 3
\[
\oint_{C}  \, \overline{z} dz = 2\pi i
\]
\begin{equation}
\oint_{C}  \, \frac{1}{z}dz=2\pi i\quad Re \left( \frac{1}{2},0\right)=2\pi i *1
\end{equation}
\begin{equation}
2\pi i
\end{equation}
Entonces \( \oint_{C}  \, \overline{z} dz=2\pi i\) 


\section{Ejercicio 5}
Sí, por el principio de deformación

\section{Ejercicio 7}
\[
f(z)=|z|\quad |z|=1
\]
\begin{equation}
|z|=\sqrt{x^2+y^2}
\end{equation}
\begin{equation}
u=\sqrt{x^2+y^2}
\end{equation}
\begin{equation}
v=0
\end{equation}
Como \( v=0\) al no haber parte imaginaria, no es analítica la función 
\begin{equation}
z=e^{it} \quad 0\leq t\leq 2\pi
\end{equation}
\begin{equation}
dz=e^{it}idt \quad 0\leq t\leq 2\pi
\end{equation}
\begin{equation}
|z|=|e^{it}|=1
\end{equation}
\begin{equation}
\int_{0}^{2\pi}e^{it}idt =i\int_{0}^{2\pi}e^{it}dt
\end{equation}
\begin{equation}
u=it \quad du= idt \quad \frac{du}{i}=dt
\end{equation}
\begin{equation}
\frac{i}{i}\int_{0}^{2\pi}e^{u}du= \left.e^u\right|_{0}^{2\pi}
\end{equation}
\begin{equation}
e^{2\pi i}-e^0=1-1=0
\end{equation}
0, no es aplicable Teorema dde Cauchy


\section{Ejercicio 9}

\[
f(z)= Im(z) \quad |z|=1
\]
\begin{equation}
Im(x+iy)=y
\end{equation}
\begin{equation}
v=0 \quad v=y
\end{equation}
Como \( v=0\) no hay parte real, por tanto la función no es analítica y el teorema de Cauchy no es aplicable
\begin{equation}
z=e^{it}t \quad 0\leq t\leq 2\pi
\end{equation}
\begin{equation}
dz=e^{it}i
\end{equation}
\begin{equation}
Im(z)= cos(t)+i sen(t)
\end{equation}
\begin{equation}
Im(z)=  sen(t)
\end{equation}
\begin{equation}
\int_{0}^{2\pi}sen(t)e^{it}idt =i\int_{0}^{2\pi}sen(t)e^{it}dt
\end{equation}
\begin{equation}
i\left[ \frac{1}{4}\left( -4cos(t) (cos(t)+isen(t))-2sen^2(t)+i(sen(2t)+2t)\right)\right]_0^{2\pi}
\end{equation}
\begin{equation}
=i\left[ \left( \frac{1}{4}\left( -4cos(2\pi) (cos(2\pi)+isen(2\pi))-2sen^2(2\pi)+isen(2\pi)+2\pi \right) \right)- \right.
\end{equation}
\begin{equation}
\left. \left(\frac{1}{4}\left( -4cos(0) (cos(0)+isen(0))-2sen^2(0)+isen(0)+2(0)\right)\right)\right]
\end{equation}
\begin{equation}
i[i\pi]=-\pi
\end{equation}
\( -\pi\). no es aplicable Teorema de Cauchy 


\section{Ejercicio 11}
\[
f(z)= \frac{1}{\overline{z}} \quad |z|=1
\]
\begin{equation}
\frac{1}{\overline{z}}=\frac{1}{x-iy}=\frac{x+iy}{x^2+y^2}
\end{equation}
\begin{equation}
u=\frac{x}{x^2+y^2} \quad v\frac{y}{x^2+y^2}
\end{equation}
\begin{equation}
\frac{\partial u}{\partial x}= \frac{y^2-x^2}{(x^2+y^2)^2} \neq \frac{\partial v}{\partial y}= \frac{x^2-y^2}{(x^2+y^2)^2}
\end{equation}
Por lo tannto, la función no es analítica y el Teorema de Cauchy no es aplicable
\begin{equation}
z=e^{it} \quad 0\leq t\leq 2\pi
\end{equation}
\begin{equation}
z=cos(t)+isen(t)
\end{equation}
\begin{equation}
\overline{z}=cos(t)-isen(t)=e^{-it}
\end{equation}
\begin{equation}
dz=e^{it}i
\end{equation}
\begin{equation}
i \int_{0}^{2\pi}\frac{1}{e^{-it}}e^{it}dt = i \int_{0}^{2\pi}\frac{e^{it}}{e^{-it}}dt=i \int_{0}^{2\pi} e^{2it}dt
\end{equation}
\begin{equation}
u=2it \quad du=2idt\quad dt=\frac{du}{2i}
\end{equation}
\begin{equation}
\frac{i}{2i}\int_{0}^{2\pi} e^udu= \left.\frac{1}{2} e^{2it}\right|_0^{2\pi}
\end{equation}
\begin{equation}
\frac{1}{2}e^{4\pi}-\frac{1}{2}e^0=\frac{1}{2}-\frac{1}{2}=0
\end{equation}
0, no se puede resolver por Teorema de Cauchy



\section{Ejercicio 13}
\[
f(z)= tan(z) \quad |z|=1
\]
\begin{equation}
tan(x+iy)=\frac{sen(z)}{cos(z)}
\end{equation}
Es analítica menos cuando cos(z)=0
\( z=\frac{\pi}{2}+k\pi\) es aplicable el teorema de Cauchy 
\begin{equation}
z=e^{it}
\end{equation}
\begin{equation}
\int \frac{f(z)}{z-z_0}dz=2\pi i f(z_0)
\end{equation}
Dado que tan(z) no tiene singularidades en su contorno en la circunferencia unitaria, es 0, por lo tanto es aplicable Teorema de Cauchy

\section{Ejercicio 15}
\[
f(z)= {\overline{z}} \quad |z|=1
\]
\begin{equation}
(x-iy)^2=x^2-2xyi-y^2
\end{equation}
\begin{equation}
u=x^2-y^2 \quad v=-2xy
\end{equation}
\begin{equation}
\frac{\partial u}{\partial x}= 2x \quad \frac{\partial v}{\partial y}= -2x
\end{equation}
\begin{equation}
\frac{\partial u}{\partial y}= -2y \quad \frac{\partial v}{\partial x}=-( -2y)=2y
\end{equation}
\begin{equation}
2x \neq -2x
\end{equation}
\begin{equation}
-2y \neq 2y
\end{equation}
Por tanto no es aplicable Teorema de Cauchy
\begin{equation}
z=e^{it} \quad 0\leq t\leq 2\pi
\end{equation}
\begin{equation}
dz=e^{it}i dt
\end{equation}
\begin{equation}
\overline{z}^2=(e^{it})^2=e^{-2it}
\end{equation}
\begin{equation}
\int_{0}^{2\pi}e^{-2it}e^{it}idt= i\int_{0}^{2\pi}e^{-it}dt
\end{equation}
\begin{equation}
u=-it \quad du=-idt \quad \frac{du}{-i}=dt
\end{equation}
\begin{equation}
\frac{i}{-i}\int_{0}^{2\pi} e^udu= \left.-e^{it}\right|_0^{2\pi}
\end{equation}
\begin{equation}
-e^{2\pi i}+e^0=-1+1=0
\end{equation}
0, no es aplicable Teorema de Cauchy


\section{Ejercicio 17}
\[
f(z)= \frac{1}{(z^2+2)} \quad |z|=1
\]
\begin{equation}
\oint_{C}  \, \frac{dz}{z^2+2}
\end{equation}
\begin{equation}
z^2+2=0 \quad z^2=-2 \quad z=\pm \sqrt{-2}
\end{equation}
\begin{equation}
z_1=i\sqrt{2} \quad z_2=-i\sqrt{2}
\end{equation}
\begin{equation}
\oint_{C}  \, \frac{dz}{(z+i\sqrt{2})(z-i\sqrt{2})}
\end{equation}
\begin{equation}
\oint_{C_1}  \, \frac{\frac{1}{z-i\sqrt{2}}}{(z+i\sqrt{2})} dz +\oint_{C_2}  \, \frac{\frac{1}{z+i\sqrt{2}}}{(z-i\sqrt{2})} dz
\end{equation}
\begin{equation}
z_0=-i\sqrt{2} \quad z_0=i\sqrt{2}
\end{equation}
\begin{equation}
\frac{1}{-i\sqrt{2}-i\sqrt{2}}=\frac{1}{-2i\sqrt{2}}\quad \frac{1}{i\sqrt{2}+i\sqrt{2}}=\frac{1}{2\sqrt{2}}
\end{equation}
\begin{equation}
\frac{-2\pi i}{2i\sqrt{2}}+\frac{2\pi i}{2i\sqrt{2}}=0
\end{equation}
0. si es aplicable Teorema de Cauchy

\section{Ejercicio 19}
\[
\oint_{C}  \, \frac{dz}{z-i} \quad |z|=2 \quad z=i=9
\]
\begin{equation}
\oint_{C}  \, \frac{f(z)}{z-9}dz=2\pi i(1)
\end{equation}
\begin{equation}
f(z)=1=2\pi i(1)=2\pi i
\end{equation}


\section{Ejercicio 21}
\[
\oint_{C}  \, \frac{cos(z)}{z}dz \quad C \quad consta \quad de \quad |z|=1 \quad |z|=3
\]
\begin{equation}
z=0 \quad \frac{1}{z}=1
\end{equation}
\begin{equation}
\oint_{C}  \, \frac{cos(z)}{z}=1
\end{equation}
Por lo tanto \( 2\pi i (1)\), lo mismo para \(|z|=3\), pero al ser contrario a las manecillas del reloj, da \(-2\pi i\)
\begin{equation}
2\pi i-2\pi i=0
\end{equation}

\section{Ejercicio 23}
\[
\oint_{C}  \, \frac{dz}{z^2-1} 
\]
\(\oint_{C}  \, \frac{dz}{z^2-1}\) por Teorema de Cauchy de curvas cerradas
\begin{equation}
2\pi if(z)=2\pi i(1)=2\pi i
\end{equation}
\begin{equation}
f(z)=1
\end{equation}
\section{Ejercicio 25}
\[
\oint_{C}  \, \frac{dz}{z^2+1} 
\]
\begin{equation}
C=a)|z+i|=1 \quad C=b)|z-i|=1
\end{equation}
\begin{equation}
\oint_{C}  \, \frac{f(z)}{z-z_0}=2\pi i f(z_0)
\end{equation}
\begin{equation}
z^2+1=0 \quad z=+i \quad z=-i
\end{equation}
\begin{equation}
z^2=-1
\end{equation}
\begin{equation}
z=\pm\sqrt{-1}
\end{equation}
\begin{equation}
\oint_{C}  \, \frac{dz}{(z-i)(z+i)}
\end{equation}
a)
\begin{equation}
=\oint_{C}  \, \frac{\frac{1}{z+i}}{(z-i)}
\end{equation}
z=i
\begin{equation}
\frac{1}{i+i} = \frac{1}{2i}
\end{equation}
\begin{equation}
2\pi i\left(\frac{1}{2i} \right)=\pi
\end{equation}
b)
\begin{equation}
=\oint_{C}  \, \frac{\frac{1}{z-i}}{(z+i)}
\end{equation}
z=-i
\begin{equation}
\frac{1}{-i-i} = -\frac{1}{2i}
\end{equation}
\begin{equation}
2\pi i\left(-\frac{1}{2i} \right)=-\pi
\end{equation}
\begin{equation}
\pi=-\pi
\end{equation}

\section{Ejercicio 27}
\[
\oint_{C}  \, \frac{2z+1}{z^2+z} 
\]
\begin{equation}
a) |z|=\frac{1}{4} \quad b) |z-\frac{1}{2}|=\frac{1}{4} \quad a) |z|=2 
\end{equation}
\begin{equation}
\oint_{C}  \, \frac{2z+1}{z(z+1)} 
\end{equation}
z=0,z=-1
\begin{equation}
\int \frac{\frac{2z+1}{z+1}}{z} dz 
\end{equation}
a) Sentido de las manecillas del reloj
\begin{equation}
2\pi i(1)=2\pi i
\end{equation}
z=0
\begin{equation}
f(z)=\frac{1}{1}
\end{equation}
\begin{equation}
f(0)=\frac{1}{1}=1
\end{equation}
b)
Las sigularidades están afuera, por lo tanto, es 0por teorema de Cauchy
c)
\begin{equation}
\int  \, \frac{\frac{2z+1}{z}}{z+1} dz 
\end{equation}
z=-1
\begin{equation}
f(-1)=\frac{2(-1)+1}{-1}=\frac{-2+1}{-1}=\frac{-1}{-1}=1
\end{equation}
\begin{equation}
2\pi i(1)=2 \pi i +2\pi i + 2\pi i = 4\pi i 
\end{equation}
Sentido de las manecillas del reloj, entonces \(-4\pi\), así: 
\begin{equation}
-2\pi,0,-4\pi i 
\end{equation}

\section{Ejercicio 29}
\[
\oint_{C}  \, \frac{3z+1}{z^3+z} dz
\]
\begin{equation}
C=a)|z|=\frac{1}{2} \quad C=b)|z|=2
\end{equation}

\begin{equation}
\int \frac{3z+1}{z(z-1)(z+1)}
\end{equation}
\begin{equation}
z=0 \quad z=1 \quad z=-1
\end{equation}
\begin{equation}
\int \frac{\frac{3z+1}{(z-1)(z+1)}}{z}
\end{equation}
a) z=0
\begin{equation}
f(0)=\frac{3(0)+1}{(0-1)(0+1)}=-\frac{1}{1}=-1
\end{equation}
\begin{equation}
=2\pi i(-1)= 2\pi i
\end{equation}
z=-1
\begin{equation}
\int \frac{\frac{3z+1}{z(z-1)}}{z+1}
\end{equation}
\begin{equation}
\frac{3(-1)+1}{-1(-1-1)}=\frac{-3+1}{2}=\frac{-2}{2}=-1
\end{equation}
\begin{equation}
=2\pi i(-1)= -2\pi i
\end{equation}
z=1
\begin{equation}
\int \frac{\frac{3z+1}{z(z+1)}}{z-1}
\end{equation}
\begin{equation}
\frac{3(1)+1}{1(1+1)}=\frac{3+1}{1(2)}=\frac{4}{2}=2
\end{equation}
\begin{equation}
=2\pi i (2)= 4\pi i
\end{equation}
\begin{equation}
=4\pi i -2\pi i -2 \pi i= 4\pi i -4 \pi i
\end{equation}
\begin{equation}
=0
\end{equation}
b)
\begin{equation}
-2 \pi i,0
\end{equation}


\end{document}