\documentclass{article}
\usepackage{graphicx}
\usepackage{amsmath}
\usepackage{amssymb}

\begin{document}
\begin{titlepage}
    \centering
    \vspace*{2cm}
    \vspace{1cm}
    {\Huge Instituto Politécnico Nacional} 
    \vspace{0.3cm}

    
    {\Large Escuela Superior de Cómputo} 
    \vspace{3cm}
    
    {\Huge \textbf{Matematicas Avanzadas para la Ingenieria}} 
    \vspace{2cm}
    
    {\Large Ejercicios Propuestos 21 de Marzo}
    \vspace{0.5cm}
    
    {\Large Integrantes:}\\
    \vspace{0.3cm}
    {\large 
    Vazquez Blancas Cesar Said\\
    
    \vfill
    
    {\large 3 de Abril de 2024}
\end{titlepage}
\newpage 

\section*{1.- Realice las operaciones indicadas }
\newline
\[
a) \frac{1}{i}
\]
\newline
1.- Calculamos
\[
0-1i
\]
2.-Solucion Calculada
\[
-i
\]
\newline
\[
b) \frac{1-i}{1+i}
\]
\newline
1.- Calculamos
\[
\dfrac{1-i}{1+i}
\]
2.-Multiplicar
Numerador y denominador
Conjugar
\[
\overline{1+i}=1-i
\]
3.-con
\[
\left(1-i\right)\,\left(1+i\right)=2
\]
\[
\dfrac{\left(1-i\right)^{2}}{2}
\]
4.-Expandiendo los paréntesis
\[
\dfrac{-2\,i}{2}
\]
5.-Solucion Calculada
\[
-i
\]
\newline
\[
c)\dfrac{2}{1-3\,i}
\]
\newline
1.- Calculamos
\[
\dfrac{2}{1-3\,i}
\]
2.-Multiplicar
Numerador y denominador
Conjugar
\[
\overline{1-3\,i}=1+3\,i
\]
3.-Expandiendo parentesis
\[
2\,\left(-4+2\,i\right)+12-16\,i
\]
con
\[
\left(1-3\,i\right)\,\left(1+3\,i\right)=10
\]
\[
\dfrac{1+3\,i}{5}
\]
4.-Solucion Calculada
\[
\dfrac{1}{5}+\dfrac{3\,i}{5}
\]
\newline
\[
d)\left(1-\sqrt{3}\,i\right)^{3}
\]
\newline
1.- Calculamos
\[
\left(1-\sqrt{3}\,i\right)^{3}
\]
2.-Elevar a una potencia
\[
\left(1-\sqrt{3}\,i\right)^{2}\,\left(1-\sqrt{3}\,i\right)
\]
\[
\left(-2-2\,\sqrt{3}\,i\right)\,\left(1-\sqrt{3}\,i\right)
\]
3.-Solucion Calculada
\[
-8
\]
\section*{2.- Encuentre la parte Real e imaginaria }
\newline
\begin{align*}
(1 + e)^{-1} &= \frac{1}{1 + e} \\
&= \frac{1}{1 + e^{i\theta}} \\
&= \frac{1}{1 + \cos(\theta) + i \sin(\theta)} \\
&= \frac{1}{1 + \cos(\theta) + i \sin(\theta)} \cdot \frac{1 + \cos(\theta) - i \sin(\theta)}{1 +\cos(\theta) - i \sin(\theta)} \\
&= \frac{1 + \cos(\theta) - i \sin(\theta)}{(1 + \cos(\theta))^2 - \sin^2(\theta)}
\end{align*}
\newpage
\section*{3.- Obtengase }
\newline
\[
a) \left(1+i\right)^{16}
\]
\newline
\begin{align*}
\left(1+i\right)^{16} &= \left(\sqrt{2} \left(\cos\left(\frac{\pi}{4}\right) + i\sin\left(\frac{\pi}{4}\right)\right)\right)^{16} \\
&= \left(\sqrt{2}\right)^{16} \left(\cos\left(16\cdot\frac{\pi}{4}\right) + i\sin\left(16\cdot\frac{\pi}{4}\right)\right) \\
&= 2^8 \left(\cos(4\pi) + i\sin(4\pi)\right) \\
&= 256 \left(\cos(0) + i\sin(0)\right) \\
&= 256
\end{align*}
\newline
\[
b) \sum_{n=0}^{100} i^n
\]
\newline
\begin{align*}
\sum_{n=0}^{100} i^n &= \underbrace{(1 + i + (-1) + (-i)) + (1 + i + (-1) + (-i)) + \dots}_{\text{25 ciclos completos}} + i^{100} \\
&= 0 + 1 \\
&= 1
\end{align*}
\newline
\[
c)\left(2+2\,\sqrt{3}\,i\right)^{9}
\]
\begin{align*}
\left(2+2\sqrt{3}i\right)^{9} &= \left(4\left(\cos\left(\frac{\pi}{3}\right) + i\sin\left(\frac{\pi}{3}\right)\right)\right)^{9} \\
&= 4^9 \left(\cos\left(9\cdot\frac{\pi}{3}\right) + i\sin\left(9\cdot\frac{\pi}{3}\right)\right) \\
&= 262144 \left(\cos(3\pi) + i\sin(3\pi)\right) \\
&= 262144 \left(-1 + i \cdot 0\right) \\
&= -262144
\end{align*}
\newpage

\section*{4.- Obtenga el módulo y el Argumento de cada uno de los siguientes complejos: }
\[
a) 3i
\]
\begin{align*}
&\text{1. Cálculo del módulo (\(r\)):} \\
&\text{El módulo (\(r\)) se calcula como la magnitud del número complejo, que es la distancia del origen al punto que representa el número en el plano complejo.} \\
&\text{Para } 3i, \text{ el módulo (\(r\)) es simplemente el coeficiente del término \(i\), que es } 3. \\
& \\
&\text{2. Cálculo del argumento (\(\theta\)):} \\
&\text{El argumento (\(\theta\)) se calcula como el ángulo que el vector del número complejo forma con el eje positivo de las \(x\) en sentido antihorario.} \\
&\text{Para } 3i, \text{ el argumento (\(\theta\)) es el ángulo cuya función trigonométrica del seno es \(1\) y del coseno es \(0\), que es } \frac{\pi}{2}. \\
& \\
&\text{Por lo tanto, el módulo (\(r\)) es } 3 \text{ y el argumento (\(\theta\)) es } \frac{\pi}{2}.
\end{align*}
\[
b) -2
\]
\begin{align*}
&\text{1. Cálculo del módulo (\(r\)):} \\
&\text{El módulo (\(r\)) se calcula como la magnitud del número complejo, que es la distancia del origen al punto que representa el número en el plano complejo.} \\
&\text{Para } -2, \text{ el módulo (\(r\)) es simplemente el valor absoluto del número real, que es } 2. \\
& \\
&\text{2. Cálculo del argumento (\(\theta\)):} \\
&\text{El argumento (\(\theta\)) se calcula como el ángulo que el vector del número complejo forma con el eje positivo de las \(x\) en sentido antihorario.} \\
&\text{Para } -2, \text{ el argumento (\(\theta\)) es el ángulo cuyo coseno es } -1 \text{ y el seno es } 0, \text{ que es } \pi. \\
& \\
&\text{Por lo tanto, el módulo (\(r\)) es } 2 \text{ y el argumento (\(\theta\)) es } \pi.
\end{align*}
\newpage
\[
c) 1+i
\]
\begin{align*}
&\text{- Cálculo del módulo (\(r\)):} \\
&\text{El módulo (\(r\)) se calcula como la magnitud del número complejo, que es la distancia del origen al punto que representa el número en el plano complejo.} \\
&\text{Para } 1+i, \text{ el módulo (\(r\)) se puede calcular utilizando el teorema de Pitágoras en el triángulo rectángulo formado por el eje real,} \\
&\text{el eje imaginario y la hipotenusa, que es la distancia desde el origen hasta el punto } 1+i \text{ en el plano complejo.} \\
&\text{Entonces, tenemos:} \\
& r = |1+i| = \sqrt{1^2 + 1^2} = \sqrt{2} \\
& \\
&\text{- Cálculo del argumento (\(\theta\)):} \\
&\text{El argumento (\(\theta\)) se calcula como el ángulo que el vector del número complejo forma con el eje positivo de las \(x\) en sentido antihorario.} \\
&\text{Para } 1+i, \text{ podemos usar las funciones trigonométricas para calcular el argumento. Dado que } 1+i \text{ está en el primer cuadrante,} \\
&\text{el argumento (\(\theta\)) es simplemente el ángulo cuya tangente es } 1, \text{ que es } \frac{\pi}{4}. \\
& \\
\end{align*}
\[
d) -1-i
\]
\begin{align*}
&\text{- Cálculo del módulo (\(r\)):} \\
&\text{El módulo (\(r\)) se calcula como la distancia del número complejo al origen en el plano complejo.} \\
&\text{Utilizamos la fórmula del módulo:} \\
& r = |z| = \sqrt{(-1)^2 + (-1)^2} = \sqrt{2} \\
& \\
&\text{- Cálculo del argumento (\(\theta\)):} \\
&\text{El argumento (\(\theta\)) se calcula como el ángulo que el número complejo forma con el eje positivo de las \(x\) en sentido antihorario.} \\
&\text{Usando trigonometría, podemos calcular el ángulo cuya tangente es \(-1\). Esto nos da un ángulo de \(-\frac{3\pi}{4}\),} \\
&\text{pero como estamos en el tercer cuadrante, sumamos \(2\pi\) para obtener el valor final del argumento:} \\
& \theta = -\frac{3\pi}{4} + 2\pi = \frac{5\pi}{4} \\
& \\
\end{align*}
\newpage
\[
e) 2+5i
\]
\begin{align*}
&\text{- Cálculo del módulo (\(r\)):} \\
&\text{El módulo (\(r\)) se calcula como la distancia del número complejo al origen en el plano complejo.} \\
&\text{Utilizamos la fórmula del módulo:} \\
& r = |z| = \sqrt{2^2 + 5^2} = \sqrt{29} \\
& \\
&\text{- Cálculo del argumento (\(\theta\)):} \\
&\text{El argumento (\(\theta\)) se calcula como el ángulo que el número complejo forma con el eje positivo de las \(x\) en sentido antihorario.} \\
&\text{Usando trigonometría, podemos calcular el ángulo cuya tangente es \(\frac{5}{2}\).  Esto nos da un ángulo de aproximadamente \(1.19029\) radianes.} \\
& \\
\end{align*}
\[
f) 2-5i
\]
\begin{align*}
&\text{- Cálculo del módulo (\(r\)):} \\
&\text{Utilizamos la fórmula del módulo:} \\
& r = |z| = \sqrt{2^2 + (-5)^2} = \sqrt{29} \\
& \\
&\text{- Cálculo del argumento (\(\theta\)):} \\
&\text{Usamos la fórmula de la tangente inversa para calcular el ángulo cuya tangente es \(-\frac{5}{2}\).} \\
&\text{Dado que el número complejo está en el cuarto cuadrante, sumamos \(2\pi\) para obtener el ángulo final:} \\
& \theta = \arctan\left(\frac{-5}{2}\right) + 2\pi \\
& \\
\end{align*}
\newpage
\[
g) -2+5i
\]
\begin{align*}
&\text{- Cálculo del módulo (\(r\)):} \\
&\text{Utilizamos la fórmula del módulo:} \\
& r = |z| = \sqrt{(-2)^2 + 5^2} = \sqrt{29} \\
& \\
&\text{- Cálculo del argumento (\(\theta\)):} \\
&\text{Usamos la fórmula de la tangente inversa para calcular el ángulo cuya tangente es \(\frac{5}{-2}\).} \\
&\text{Dado que el número complejo está en el segundo cuadrante, sumamos \(\pi\) para obtener el ángulo final:} \\
& \theta = \arctan\left(\frac{5}{-2}\right) + \pi \\
& \\
\end{align*}
\[
h) -2-5i
\]
\begin{align*}
&\text{- Cálculo del módulo (\(r\)):} \\
&\text{Utilizamos la fórmula del módulo:} \\
& r = |z| = \sqrt{(-2)^2 + (-5)^2} = \sqrt{29} \\
& \\
&\text{- Cálculo del argumento (\(\theta\)):} \\
&\text{Usamos la fórmula de la tangente inversa para calcular el ángulo cuya tangente es \(\frac{-5}{-2}\).} \\
&\text{Dado que el número complejo está en el tercer cuadrante, sumamos \(\pi\) para obtener el ángulo final:} \\
& \theta = \arctan\left(\frac{-5}{-2}\right) + \pi \\
& \\
\end{align*}
\newpage
\[
i) bi, b diferente de 0
\]
\begin{align*}
&\text{- Cálculo del módulo (\(r\)):} \\
&\text{El módulo (\(r\)) de un número complejo en el eje imaginario es simplemente el valor absoluto de la parte imaginaria,} \\
&\text{es decir, } r = |bi| = |b| = b. \\
& \\
&\text{- Cálculo del argumento (\(\theta\)):} \\
&\text{Dado que el número complejo } bi \text{ se encuentra en el eje imaginario, su argumento (\(\theta\)) es} \\
&\frac{\pi}{2} \text{ si } b > 0 \text{ y } -\frac{\pi}{2} \text{ si } b < 0. \\
& \\
\end{align*}
\[
j) a+bi, a diferente de 0
\]
\begin{align*}
&\text{1. Cálculo del módulo (\(r\)):} \\
&\text{El módulo } r \text{ de un número complejo } z = a + bi \text{ se calcula como:} \\
& r = |z| = \sqrt{a^2 + b^2} \\
& \\
&\text{2. Cálculo del argumento (\(\theta\)):} \\
&\text{El argumento } \theta \text{ de un número complejo } z = a + bi \text{ se calcula como:} \\
& \theta = \operatorname{atan2}(b, a) \\
& \\
\end{align*}
\newpage
\section*{6.- Encuéntrense fórmulas para sumar las expresiones siguientes: }
\[
a) 1 + cos θ + cos 2θ + cos 3θ + · · · + cos nθ
\]
\begin{align*}
&\text{1. Aplicación de la fórmula de la suma de una serie finita de cosenos:} \\
&\text{La fórmula de la suma de una serie finita de cosenos es:} \\
&\sum_{k=0}^{n} \cos(k\theta) = \frac{\sin\left(\frac{(n+1)\theta}{2}\right)}{\sin\left(\frac{\theta}{2}\right)} \cdot \cos\left(\frac{n\theta}{2}\right) \\
& \\
&\text{2. Sustitución en la serie dada:} \\
&\text{La serie dada es } 1 + \cos(\theta) + \cos(2\theta) + \cos(3\theta) + \ldots + \cos(n\theta) \\
&\text{Por lo tanto, aplicamos la fórmula de la suma de cosenos a partir de } k = 0 \text{ hasta } k = n: \\
& 1 + \sum_{k=1}^{n} \cos(k\theta) = 1 + \frac{\sin\left(\frac{(n+1)\theta}{2}\right)}{\sin\left(\frac{\theta}{2}\right)} \cdot \cos\left(\frac{n\theta}{2}\right) \\
& \\
&\text{3. Expresión final:} \\
&\text{La expresión final de la suma de la serie es:} \\
&  \frac{\sin\left(\frac{(n+1)\theta}{2}\right)}{\sin\left(\frac{\theta}{2}\right)} \cdot \cos\left(\frac{n\theta}{2}\right)
\end{align*}
\newpage
\[
b) sen θ + sen 2θ + sen 3θ + · · · + sen nθ
\]
\begin{align*}
&\text{1. Aplicación de la fórmula de la suma de una serie finita de senos:} \\
&\text{La fórmula de la suma de una serie finita de senos es:} \\
&\sum_{k=0}^{n} \sin(k\theta) = \frac{\sin\left(\frac{(n+1)\theta}{2}\right) \cdot \sin\left(\frac{n\theta}{2}\right)}{\sin\left(\frac{\theta}{2}\right)} \\
& \\
&\text{2. Sustitución en la serie dada:} \\
&\text{La serie dada es } \sin(\theta) + \sin(2\theta) + \sin(3\theta) + \ldots + \sin(n\theta) \\
&\text{Por lo tanto, aplicamos la fórmula de la suma de senos a partir de } k = 0 \text{ hasta } k = n: \\
& \sin(\theta) + \sum_{k=1}^{n} \sin(k\theta) = \sin(\theta) + \frac{\sin\left(\frac{(n+1)\theta}{2}\right) \cdot \sin\left(\frac{n\theta}{2}\right)}{\sin\left(\frac{\theta}{2}\right)} \\
& \\
&\text{3. Expresión final:} \\
&\text{La expresión final de la suma de la serie es:} \\
&  \frac{\sin\left(\frac{(n+1)\theta}{2}\right) \cdot \sin\left(\frac{n\theta}{2}\right)}{\sin\left(\frac{\theta}{2}\right)}
\end{align*}
\[
c) cos θ + cos 3θ + · · · + cos 2(n + 1)θ
\]
\begin{align*}
&\text{1. Aplicación de la fórmula de la suma de una serie finita de cosenos:} \\
&\text{La fórmula de la suma de una serie finita de cosenos es:} \\
&\cos(\theta) + \cos(3\theta) + \ldots + \cos(2(n + 1)\theta) = \frac{\sin\left((2n + 2)\theta/2\right) \cdot \cos(\theta/2)}{\sin(\theta/2)} \\
& \\
&\text{2. Sustitución en la serie dada:} \\
&\text{La serie dada es } \cos(\theta) + \cos(3\theta) + \ldots + \cos(2(n + 1)\theta) \\
&\text{Por lo tanto, aplicamos la fórmula de la suma de cosenos:} \\
&\cos(\theta) + \cos(3\theta) + \ldots + \cos(2(n + 1)\theta) = \frac{\sin\left(2\theta\right) +(\sin\left(4\theta\right)-\sin\left(2\theta\right))+(\sin\left(6\theta\right)-\sin\left(4\theta\right))}{2\sin\theta} \\
& \\
&\text{3. Expresión final:} \\
&\text{La expresión final de la suma de la serie es:} \\
&\frac{\sin\left(2n\theta\right) \cdot}{2\sin\theta}
\end{align*}
\section*{9.- Sean A, B, C, D números reales sujetos a la condición A2 + B2 + C
2 >

D2
. Pruébense que:
a) la ecuación
A (z + z) + iB (z − z) + C (zz − 1) + D (zz + 1) = 0
representa: una recta en el plano, caso de que C + D = 0; una
circunferencia, de centro y radio a determinar, en el caso de que
C + D ̸= 0. (Circunrecta de parámetros reales A, B, C, D, de
ahora en adelante.)
b) toda circunrecta en el plano responde a una ecuación de la forma
dada arriba, para convenientes números reales A, B, C, D. }
\textbf{Parte a)}

Consideremos la ecuación:
\[ A (z + \overline{z}) + iB (z - \overline{z}) + C (zz - 1) + D (zz + 1) = 0 \]

Donde \( z = x + iy \) y \( \overline{z} = x - iy \).

Reorganizando, obtenemos:
\[ (A + C) x + i(B - A) y + (D - C) x^2 + (D + C) y^2 - C - D = 0 \]

Si \( C + D = 0 \), la ecuación representa una recta en el plano.

Si \( C + D \neq 0 \), la ecuación representa una circunferencia de la forma:
\[ (D - C) u^2 + (A + C) u + (D + C) v^2 + i(B - A) v = C + D \]

\textbf{Parte b)}

Toda circunrecta en el plano complejo puede representarse mediante una ecuación de la forma:
\[ (D - C) u^2 + (A + C) u + (D + C) v^2 + i(B - A) v = C + D \]
para convenientes números reales \( A, B, C, D \).
\newpage
\[
b) sen θ + sen 2θ + sen 3θ + · · · + sen nθ
\]
\section*{Demostración de la Fórmula del Argumento Principal}

Dado un número complejo \( z = x + iy \), donde \( x \) e \( y \) son las partes real e imaginaria respectivamente, queremos probar la fórmula para el argumento principal \( \text{arg}(z) \).

\subsection*{Caso 1: \( x > 0 \)}
Si \( x > 0 \), entonces \( \text{arg}(z) = \arctan\left(\frac{y}{x}\right) \).

\subsection*{Caso 2: \( x < 0 \) y \( y \geq 0 \)}
Si \( x < 0 \) y \( y \geq 0 \), entonces \( \text{arg}(z) = \arctan\left(\frac{y}{x}\right) + \pi \).

\subsection*{Caso 3: \( x < 0 \) y \( y < 0 \)}
Si \( x < 0 \) y \( y < 0 \), entonces \( \text{arg}(z) = \arctan\left(\frac{y}{x}\right) - \pi \).

\subsection*{Caso 4: \( x = 0 \) y \( y > 0 \)}
Si \( x = 0 \) y \( y > 0 \), entonces \( \text{arg}(z) = \frac{\pi}{2} \).

\subsection*{Caso 5: \( x = 0 \) y \( y < 0 \)}
Si \( x = 0 \) y \( y < 0 \), entonces \( \text{arg}(z) = -\frac{\pi}{2} \).

\subsection*{Conclusiones}
Hemos demostrado la fórmula para el argumento principal de un número complejo en diferentes casos, cubriendo todos los posibles valores de \( x \) e \( y \).

\end{document}
\end{document}
