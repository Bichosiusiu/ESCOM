\documentclass{article}
\usepackage{graphicx}
\usepackage{amsmath}
\usepackage{amssymb}

\begin{document}
\begin{titlepage}
    \centering
    \vspace*{2cm}
    \vspace{1cm}
    {\Huge Instituto Politécnico Nacional} 
    \vspace{0.3cm}

    
    {\Large Escuela Superior de Cómputo} 
    \vspace{3cm}
    
    {\Huge \textbf{Matematicas Avanzadas para la Ingenieria}} 
    \vspace{2cm}
    
    {\Large Transformacion de Variable Compleja}
    \vspace{0.5cm}
    
    {\Large Integrantes:}\\
    \vspace{0.3cm}
    {\large Arellano Millan Gabriel \\
    Gomez Tovar Yoshua Oziel \\
    Herrera Tovar Karla Elena \\
    Vazquez Blancas Cesar Said\\
    Zarco Sosa Kevin\\ 
    
    \vfill
    
    {\large 30 de Marzo de 2024}
\end{titlepage}


\newpage 

\section*{1.- Ejercicio 1 }
\newline
Encontrar f(3 + i),f(-1),f(-4 + 2i), donde f(z) es igual a 
\[
z^2 + 2z
\]
Para  f(3 + i):
\newline
1.- Calculamos
\[
\left(3+i\right)^{2}+2\,\left(3+i\right)
\]
2.-Expandimos parentesis
\[
8+6\,i+2\,\left(3+i\right)
\]
3.-Expandiendo los paréntesis
\[
2\,\left(3+i\right)+8+6\,i
\]
\[
6+2\,i+8+6\,i
\]
4.-Solucion Calculada
\[
14+8\,i
\]
\newline
Para  f(-i):
\newline
1.- Calculamos
\[
(-i)^2 + 2(-i)
\]
2.--i elevado a 2 es -1
\[
-1 + 2(-i)
\]
3.-Solucion Calculada
\[
-1-2i
\]
\newline
Para  f(-4+2i):
\newline
1.- Calculamos
\[
\left(-4+2\,i\right)^{2}+2\,\left(-4+2\,i\right)
\]
2.-Expandimos Parentesis
\[
12-16\,i+2\,\left(-4+2\,i\right)
\]
3.-Expandiendo parentesis
\[
2\,\left(-4+2\,i\right)+12-16\,i
\]
\[
-8+4\,i+12-16\,i
\]
4.-Solucion Calculada
\[
4-12\,i
\]
\section*{2.- Ejercicio 3 }
\newline
Encontrar f(3 + i),f(-1),f(-4 + 2i), donde f(z) es igual a 
\[
\dfrac{1}{\left(z\right)^{3}}
\]
Para  f(3 + i):
\newline
1.- Calculamos
\[
\dfrac{1}{\left(3+i\right)^{3}}
\]
2.-Elevar a la potencia
\[
\dfrac{1}{\left(3+i\right)\,\left(3+i\right)^{2}}
\]
\[
\dfrac{1}{\left(3+i\right)\,\left(8+6\,i\right)}
\]
3.-Multiplicar
Numerador y denominador
Conjugar
\[
\overline{\left(3+i\right)\,\left(8+6\,i\right)}=\left(3-i\right)\,\left(8-6\,i\right)
\]
\[
\left(8-6\,i\right)\,\left(3-i\right)\,\left(3+i\right)\,\left(8+6\,i\right)=1000
\]
\[
\dfrac{\left(8-6\,i\right)\,\left(3-i\right)}{1000}
\]
4.-Expandiendo los paréntesis
\[
\dfrac{\left(8-6\,i\right)\,\left(3-i\right)}{1000}
\]
\[
\dfrac{18-26\,i}{1000}
\]
5.-La solucion es: 
\[
\dfrac{9}{500}-\dfrac{13\,i}{500}
\]
\newline
Para  f(-i):
\newline
1.- Calculamos
\[
\dfrac{1}{\left(-i\right)^{3}}
\]
2.--i elevado a 3 es i
\[
\dfrac{1}{\left(i\right)}
\]
3.-Solucion Calculada
\[
-i
\]
\newline
Para  f(-4+2i):
\newline
1.- Calculamos
\[
\dfrac{1}{\left(-4+2\,i\right)^{3}}
\]
2.-Elevar a una potencia
\[
\dfrac{1}{\left(-4+2\,i\right)^{2}\,\left(-4+2\,i\right)}
\]
\[
\dfrac{1}{\left(12-16\,i\right)\,\left(-4+2\,i\right)}
\]
3.-Multiplicar
Numerador y denominador
Conjugar
\[
\overline{\left(12-16\,i\right)\,\left(-4+2\,i\right)}=\left(12+16\,i\right)\,\left(-4-2\,i\right)
\]
\[
\left(12-16\,i\right)\,\left(\left(-4-2\,i\right)\,\left(-4+2\,i\right)\,\left(12+16\,i\right)\right)=8000
\]
\[
\dfrac{\left(-4-2\,i\right)\,\left(12+16\,i\right)}{8000}
\]
4.-Expandiendo los paréntesis
\[
\dfrac{\left(-4-2\,i\right)\,\left(12+16\,i\right)}{8000}
\]
\[
\dfrac{-16-88\,i}{8000}
\]
5.-Agrupamiento
Parte real e imaginaria
\[
-\dfrac{1}{500}-\dfrac{11\,i}{1000}
\]
\section*{3.- Ejercicio 5 }
\newline
Encontrar las partes real e imaginaria de las siguientes funciones
\[
f(z) = 2z^3 - 3z
\]
\newline
1.- Calculamos
\[
2\,\left(x+i\,y\right)^{3}-3\,\left(x+i\,y\right)
\]
2.-Elevar a la potencia
\[
2\cdot \left(x+i\,y\right)\,\left(x+i\,y\right)^{2}-3\,\left(x+i\,y\right)
\]
\[
2\,\left(x+i\,y\right)\,\left(-y^{2}+x^{2}+2\,i\,x\,y\right)-3\,\left(x+i\,y\right)
\]
3.-Expandimos parentesis
\[
2\,\left(x+i\,y\right)\,\left(-y^{2}+x^{2}+2\,i\,x\,y\right)-3\,\left(x+i\,y\right)
\]
\[
2\,\left(-3\,x\,y^{2}+x^{3}-i\,y^{3}+3\,i\,x^{2}\,y\right)-3\,x-3\,i\,y
\]
4.-Expandiendo los paréntesis
\[
2\,\left(-3\,x\,y^{2}+x^{3}-i\,y^{3}+3\,i\,x^{2}\,y\right)-3\,x-3\,i\,y
\]
\[
-6\,x\,y^{2}+2\,x^{3}-2\,i\,y^{3}+6\,i\,x^{2}\,y-3\,x-3\,i\,y
\]
5.-La solucion es: 
\[
-6\,x\,y^{2}+2\,x^{3}-3\,x+i\,\left(-2\,y^{3}+6\,x^{2}\,y-3\,y\right)
\]
\section*{4.- Ejercicio 7 }
\newline
Suponer que z varía en una región R del plano z. Encontrar la región (precisa) del plano w en
que están los valores correspondientes de w = J(z), y muestre gráficamente ambas regiones
\[
f(z)=z^2, |z|>3
\]
\newline
1.- Calculamos a |w|
\[
w= (x^2-y^2)+i(2xy)
\]
2.-Calculamos el modulo
\[
\left|-y^{2}+x^{2}+2\,i\,x\,y\right|
\]
\[
\sqrt{x^4-2x^2y^2+y^4+4x^2y^2}
\]
\[
\sqrt{x^4+2x^2y^2+y^4}
\]
\[
\sqrt{(x^2+y^2)^2}
\]
\[
|w|=x^2+y^2
\]
3.- Calculamos el modulo de z
\[
\sqrt{x^2+y^2}>3
\]
\[
x^2+y^2>3^2
\]
\[
x^2+y^2>9
\]
4.- Sustituimos w
\[
|w|>9
\]
5.- La solucion es 
\[
|w|>9
\]
\begin{figure}
    \centering
    \includegraphics[width=0.5\textwidth]{img1.jpeg} % Ajusta el tamaño según sea necesario
    \caption{circulo creado con modulo mayor al radio al cuadrado.}
    \label{fig:mi_imagen}
\end{figure}

\section*{5.- Ejercicio 9 }
\newline
Suponer que z varía en una región R del plano z. Encontrar la región (precisa) del plano w en
que están los valores correspondientes de w = J(z), y muestre gráficamente ambas regiones
\[
f(z)=z^3, |argz|<=  \pi/4
\]
\newline
\begin{figure}
    \centering
    \includegraphics[width=0.4\textwidth]{img2.jpeg} % Ajusta el tamaño según sea necesario
    \label{fig:mi_imagen}
\end{figure}
1.- Damos 3 vueltas, por lo tanto 
\[
|argw|<=\pi/4 (3)
\]
2.- La solucion es 
\[
|argw|<= 3\pi / 4
\]

\end{document}
